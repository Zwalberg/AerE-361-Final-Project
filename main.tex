\documentclass[12pt]{article}
\renewcommand{\thesection}{\Roman{section}} 
\renewcommand{\thesubsection}{\thesection.\Roman{subsection}}
%\usepackage[tocindentauto]{tocstyle}
%\usetocstyle{KOMAlike} %the previous line resets it
%\usepackage{natbib}
\usepackage{biblatex}
\addbibresource[]{ref.bib}
\usepackage{url}
\usepackage[utf8]{inputenc}
\usepackage{amsmath}
\usepackage{graphicx}
\usepackage{graphviz}
\usepackage[T1]{fontenc}
\graphicspath{{images/}}
\usepackage{parskip}
\usepackage{fancyhdr}
\usepackage{hyperref}
\usepackage{parskip}
\usepackage{hologo}
\usepackage{listings}
\usepackage{titlesec, blindtext, color}
\usepackage{titling}
\usepackage{tcolorbox}
\usepackage[hmargin=1in,vmargin=1in]{geometry}
\usepackage{float}
\usepackage{tikz}
\usepackage{appendix}
\usepackage{listings} % For code importing
\usepackage{xcolor} % for setting colors
\usepackage{svg}
\usepackage{tocloft}
\renewcommand{\cftsecleader}{\cftdotfill{\cftdotsep}}

\input{arduinoLanguage.tex}

\hypersetup{
	colorlinks=true,
	linkcolor=blue,
	urlcolor=cyan,
}

\lstdefinestyle{customc}{
  belowcaptionskip=1\baselineskip,
  breaklines=true,
  frame=L,
  xleftmargin=\parindent,
  language=C,
  showstringspaces=false,
  basicstyle=\footnotesize\ttfamily,
  keywordstyle=\bfseries\color{green!40!black},
  commentstyle=\itshape\color{purple!40!black},
  identifierstyle=\color{blue},
  stringstyle=\color{orange},
 }

 \lstset{ %
  backgroundcolor=\color{white},   % choose the background color; you must add \usepackage{color} or \usepackage{xcolor}
  basicstyle=\footnotesize,        % the size of the fonts that are used for the code
  breakatwhitespace=false,         % sets if automatic breaks should only happen at whitespace
  breaklines=true,                 % sets automatic line breaking
  captionpos=b,                    % sets the caption-position to bottom
  commentstyle=\color{commentsColor}\textit,    % comment style
  deletekeywords={...},            % if you want to delete keywords from the given language
  escapeinside={\%*}{*)},          % if you want to add LaTeX within your code
  extendedchars=true,              % lets you use non-ASCII characters; for 8-bits encodings only, does not work with UTF-8
  frame=tb,	                   	   % adds a frame around the code
  keepspaces=true,                 % keeps spaces in text, useful for keeping indentation of code (possibly needs columns=flexible)
  keywordstyle=\color{keywordsColor}\bfseries,       % keyword style
  language=Python,                 % the language of the code (can be overrided per snippet)
  otherkeywords={*,...},           % if you want to add more keywords to the set
  numbers=left,                    % where to put the line-numbers; possible values are (none, left, right)
  numbersep=8pt,                   % how far the line-numbers are from the code
  numberstyle=\tiny\color{commentsColor}, % the style that is used for the line-numbers
  rulecolor=\color{black},         % if not set, the frame-color may be changed on line-breaks within not-black text (e.g. comments (green here))
  showspaces=false,                % show spaces everywhere adding particular underscores; it overrides 'showstringspaces'
  showstringspaces=false,          % underline spaces within strings only
  showtabs=false,                  % show tabs within strings adding particular underscores
  stepnumber=1,                    % the step between two line-numbers. If it's 1, each line will be numbered
  stringstyle=\color{stringColor}, % string literal style
  tabsize=2,	                   % sets default tabsize to 2 spaces
  title=\lstname,                  % show the filename of files included with \lstinputlisting; also try caption instead of title
  columns=fixed                    % Using fixed column width (for e.g. nice alignment)
}

\lstdefinestyle{customasm}{
  belowcaptionskip=1\baselineskip,
  frame=L,
  xleftmargin=\parindent,
  language=[x86masm]Assembler,
  basicstyle=\footnotesize\ttfamily,
  commentstyle=\itshape\color{purple!40!black},
}

\lstset{escapechar=@,style=customc}

%\makeatletter
%\let\thetitle\@title

%\let\thedate\@date
%\makeatother

%\pagestyle{fancy}
%\fancyhf{}
%\rhead{\theauthor}
%\lhead{\thetitle}
%\cfoot{\thepage}

\begin{document}
\title{Project Proposal}
%%%%%%%%%%%%%%%%%%%%%%%%%%%%%%%%%%%%%%%%%%%%%%%%%%%%%%%%%%%%%%%%%%%%%%%%%%%%%%%%%%%%%%%%%

\begin{titlepage}
	\centering
    \vspace*{0.5 cm}
    \includegraphics[scale = 0.11]{isu_seal.png}\\[1.0 cm]	% University Logo %Need isu_seal.png
    \textsc{\LARGE IOWA STATE UNIVERSITY}\\[2.0 cm]
    \textsc{\large AEROSPACE ENGINEERING DEPARTMENT}\\[0.2 cm]
    \textsc{\large Computational Techniques for Aerospace Design}\\[0.2 cm]
	\textsc{\Large AERE 361}\\[0.5 cm]				% Course Code
	\textsc{\Large Project Proposal}\\[0.2 cm]
	\textsc{\Large Group 10}\\[0.2 cm]
	\rule{\linewidth}{0.2 mm} \\[0.4 cm]
	%{ \huge \bfseries \thetitle}\\
	
	
	\begin{minipage}{0.8\textwidth}
		
			\begin{flushleft} 
			\emph{Team Member Names :} \\
			Nicholas Johnson\linebreak
			Clayton McPhail\linebreak
			Jessica Melville\linebreak
			Victoria Fleming\linebreak
			Sam Cvikota\linebreak
			Zachary Walberg\linebreak
			
		\end{flushleft}
	\end{minipage}\\[2 cm]
	
	\vfill
	
\end{titlepage}

%%%%%%%%%%%%%%%%%%%%%%%%%%%%%%%%%%%%%%%%%%%%%%%%%%%%%%%%%%%%%%%%%%%%%%%%%%%%%%%%%%%%%%%%%
%\maketitle
\tableofcontents
\pagebreak
%%%%%%%%%%%%%%%%%%%%%%%%%%%%%%%%%%%%%%%%%%%%%%%%%%%%%%%%%%%%%%%%%%%%%%%%%%%%%%%%%%%%%%%%%

\section{ABSTRACT}
%The abstract is a summary of your proposal. In general, your abstract should have enough information so that if I was to copy and paste your abstract into a web site, people would get the general idea of what your proposal is about. It should not go into any heavy detail, just the basics of what your project is about. The who, the what, and the why. You should keep your abstract to 200-400 words. Use this to ``hook in'' your reader.

Blackjack, a commonly known household game, is often used to teach players another strategy: counting cards. Studies show that counting cards is an exercise in memory, mental math, and probability and statistics, and counting cards increases the player’s chance of winning. This project will entail the creation of a handheld Blackjack game. The game will consist of the display of an Adafruit CLUE board, the built-in board buttons, along with three additional buttons. Along with the basic features of Blackjack, the game will feature a card counter that can be displayed when the user requests. Through this game, the user can practice basic Blackjack strategy and counting cards, enabling players to beat casinos at their own game.

\section{INTRODUCTION}
%While the abstract and introduction may seem like it is similar, remember that your abstract should have enough information to stand on its own. The introduction is really the actual start to your proposal. Here you should introduce the project, the people involved and give a short introduction to the why you are doing this. This should be 1-3 paragraphs.

Counting cards in Blackjack is commonly frowned upon at casinos, to the point it can get a person banned. However, studies have shown that counting cards helps with mental math and memory, and is an application of statistics and probability (reference 2). While players may not plan on becoming a Blackjack expert or playing in casinos, counting cards is a strategy that can be used in a scenario as simple as family game night to improve a player’s chances of winning (reference 1). 

Technologies such as cell phones and laptops are capable of having card counters, but it would be fairly obvious if these were being used while playing Blackjack. Therefore, it is beneficial for players who wish to count cards to learn how to do so without any external assistance. This project will provide an aid in the process of learning to count cards in the form of a handheld Blackjack game. Players can practice basic strategy while playing a computer. If desired, players can count cards and check their counts by using a built-in counter displayed upon request. With enough practice, players can learn how to mentally count cards, improving their odds of winning Blackjack.

\section{FEATURES}
%Your Features section must include a listing of at least three key features that makes your project unique. Each item needs to be backed up with a description of what it will do and why. A listing of just three items is not enough, you need to describe what those features are and why your group feels they are needed. For that reason your features should have a paragraph for each key item that describes what that key feature is. A key feature should be something that is significant to your project. For example, a key feature an autopilot system is the ability to be able to set an altitude and the autopilot will automatically set the airspeed. That is a significant feature that has a large impact on that system.

The first feature of the Blackjack game are buttons to hold, hit, and split. These are essential functions, as the game cannot be played without them. Having buttons for these features is easier and more accessible for the player than having pop-up menus during the game.

The second feature utilizes the built-in buttons on the Adafruit CLUE board. These buttons will be programmed to display the card counter and end the game. The card counter will be controlled by a button so it can be displayed whenever the user wishes. This allows the user to mentally practice counting cards and check their counts whenever they desire. An end game button allows the player to end the game at any time.

The final feature will utilize the lights on the Adafruit CLUE board. The lights will show the result of the round, with green being the player winning and red with the dealer winning. This provides an easy visual aid to determine who the winner of the game is.


% Below is an example of inserting an image.  Not that LaTex
% will determine the best location for the image.  Make sure
% you replace this image with yours and place a proper caption.
% You can use the \label{name} to name the figure and then reference
% it from your writeup and LaTeX will automatically give it the correct
% number. 
%\begin{figure}[!t]
%\centering
%\includegraphics[width=4.5in]{cpx01.jpg}
%\caption{This is the Circuit Playground Express}
%\label{fig:cpx}
%\end{figure}

\section{PROBLEM STATEMENT}
%Here you will go into more detail on what problem you hope to solve or address.  You should discuss what the problem is and why it is important to solve it. In this section, you need to be clear on what the problem is, so do not think of this as a ``light'' section. It helps to define your project.

%Your team needs to do some research into the problem at hand. Becuase of that, you should have around two to three references that you are pulling from. There are lots of places you can find references from including the ISU library and Google Scholar. I would also suggest looking at Adafruit's website, as you may find inspiration or looking to improve something already there. Remember to cite your sources though. If you find something online, that can often be citation.

%When you create your ``ref.bib'' file, don't forget to follow the standards for a BiBTex file. Certain things like webistes requires certain keywords for it to render properly. There are lots of sources online to help with this and many places like the ISU Library and Google Scholar can also generate text that is compatible with a BiBTex file. Once you have your Bib file ready, don't forget to cite your citations in your proposal like this \cite{einstein} or this \cite{dirac}.

One problem among children and adults is they are not proficient in mental math. Mental math is doing arithmetic using only the brain. Mental math doesn't include a pencil and paper, a calculator, or counting fingers. This skill seems to be a dying art, as more children and adults require a calculator to do even the most simple calculations, like addition and subtraction. 
A way to practice this skill while also having fun could be the key to incentivizing all ages to practice this skill. According to EurekAlert!, one effect of practicing mental math includes bettering emotional health, which can help with depression and anxiety.

When mental math is a practiced skill, there is greater satisfaction when solving simple math calculations, and time is saved from not having to pull a calculator out. 
Moreover, sharpening mental capacity is a goal for many people. One way to sharpen mental capacity is by holding a value in memory and applying basic math operations. Repeating this process not only enhances mental math, but is a great way to “delay memory decline and fight dementia, Alzheimer’s, and other senility-related problems” (Turner, 2019).
As can be seen, mental math can have enormously positive effects on the health of your brain. 

In addition, “the most common cause of mental decline is boredom, routine, and lack of challenging activities to do” (Turner, 2019). To combat boredom, we want to devise a way to integrate practicing mental math with a stimulating game. A fun game will entice a player to play while disguising that the player is practicing their mental math skills. 

Another area that can be addressed is creating a game requiring only one player. There is simplicity in playing a game against yourself. That's why games such as Solitaire have endured for decades and are still popular to this day. Tweaking this idea a bit, we can create a game that plays against the computer, so there is no need for another player. 
Overall, we want to find a solution to enhance people's mental math capacity while allowing them to have fun while doing it. Studies show that practicing a skill using a game is more effective because “fun motivates students and helps them pay attention and stay focused on the subject”(Why use games to teach? 2018). Taking all of this into consideration, we intend to create a Blackjack game with a card counter that can be displayed by user request.

\section{PROBLEM SOLUTION}
%Here go over your approach to your solution and what your solution is. You must include at least one image that shows your concept. This image can be a sketch or drawing or some pictures that show your concept. Make sure you reference the image(s) like this - Figure \ref{fig:cpx}. Finally, make sure you replace the stock image I included. You should also reference any sources you had from your problem statement as well.

%You must also include a table that lists all the parts that you wish to have. As announced in class, you will have the parts listed in Table \ref{table:parts_list}. We have plenty of two additional parts. Those are a conductive adhesive strip and a neopixel strip. I do have some other parts, such as arcade buttons and some additional sensors. You can certainly ask for something, and I will see what I can do. Change the table below to reflect the parts you are requesting.

%\begin{table}[ht]
%  \caption{Parts available for teams}
%  \label{table:parts_list}
%  \begin{center}
%  \begin{tabular}{|p{3in}|c|}
  
%  \hline
%  Part description & Qty\\
%  \hline
%  \hline
%  Adafruit Circuit Playground Express & 1 \\
%  \hline
%  AAA Battery Holder & 1 \\
%  \hline
%  USB Cable & 1 \\
%  \hline
%  \end{tabular}
%  \end{center}
%  \end{table}

%Finally, you can also include any pseudo code or any code snippets you have gathered so far.  This is not required, but if you found some starter code or came up with some ideas for the code, put it here. If you want to embed code into \LaTeX, you can use the example below on how to do this in \LaTeX.

For the gameplay, 1 deck of cards will be stored in the system in arrays with each card being able to be pulled up to 4 times meaning the game will be played with 4 decks. A random number generator will determine which of the cards are pulled and inserted into the game. The code will check against the array of cards ensuring the chosen card has not been played 4 times up to this point. Once the card is cleared to be chosen, the count of times the card has been played in the array will increase by 1. Once the computer has dealt two cards to each player the player will be shown the total point value of their cards and the opponents. At this point the player will hit the button corresponding with their strategic choice to take another card (hit) or to keep the cards they have (stand). Once the player has chosen to stand, the dealer’s second card will be revealed. Choosing the cards to be dealt will use the same method as the initial deal for each card and the point total will be summed each time. The choice of cards will most likely be dealt with using a \texttt{do} loop so that the process will continue until either bust or stand with \texttt{if} statements inside for each player choice. If the player’s cards total over 21 the loop would immediately exit with a message telling the player they busted illuminating the correct light. Once the player stands the computer will make the same decisions, abiding by a set of rules similar to those a casino dealer would follow. The outcome of the game would then be shown by the comparison of the computers point tally versus the players point tally. Should the player be dealt an ace and a card with a point value of 10 the game will tell the player they have blackjack and the hand will be over with protocol for winning followed.

The true purpose of this game is to allow players to test their skills at counting cards. When the trigger button is hit the game will display the current card count. The count will be determined using a simple system. Inside the array will also be the count value of the card. The method we chose to use divides the deck into three sections, low cards with a point value of 6 or less will  be assigned a +1 value, mid range cars with point values of 7-9 will be assigned a 0 value, and cards with a point value of 10, also including aces, will be assigned a -1 value. This is shown in 
Table \ref{table:card_points}. 

\begin{table}[ht]
\begin{tabular}{|l|l|l|l|}
\hline
\textbf{Card} & \textbf{Point Value} & \textbf{Count Value} \\ \hline
2 & 2 & +1 \\ \hline
3 & 3 & +1 \\ \hline
4 & 4 & +1 \\ \hline
5 & 5 & +1 \\ \hline
6 & 6 & +1 \\ \hline
7 & 7 & 0 \\ \hline
8 & 8 & 0 \\ \hline
9 & 9 & 0 \\ \hline
10 & 10 & +1 \\ \hline
Jack & 10 & +1 \\ \hline
Queen & 10 & +1 \\ \hline
King & 10 & +1 \\ \hline
Ace & 1 or 11 & +1 \\ \hline
\end{tabular}
\caption{Card Point and Count Values}
\label{table:card_points}
\end{table}

The computer count will be updated when each card is chosen through the process described previously. This count has no effect on the player or the way the game progresses–as far as the card choosing process is concerned. The count would only be reset once the number of cards that could be dealt reaches a predetermined minimum, or the player manually resets it reshuffling the decks.

We chose to allow each card to be chosen 4 times–essentially playing with 4 decks–because 4 decks makes it harder for players to count individual cards and increases the complexity of the probability calculation and encourages the use of our methods. It is also closer to the procedures at real casinos where they use multiple decks. The player will be playing against a computer dealer where we will program in casino rules such as holding on 17 and hitting on 16. 

\begin{figure}[ht]
\centering
\includegraphics[width=3in]{IMAGE ADDRESS}  %NEED IMAGE ADDRESS FROM PROJECT PROPOSAL
%IMAGE DESCRIPTION AdaFruit display with output
\caption{Concept mockup of the Adafruit CLUE Board screen displaying our Blackjack game.}
\label{fig:blackjack1}
\end{figure}

Figure \ref{fig:blackjack1} above shows a concept of the clue board’s display while the program is running. This is an example of what the display would show at the end of a game. The box marked with the red one is where information will be relayed to the user. This information could include their decision to “Hit”, “Stand”, or “Split”. This area could also display any user error. For example, if the split button is used incorrectly, there could be an error message displayed. Since the figure above shows what would be displayed at the end of the game, an example of the display at the beginning of the game can be seen below in Figure \ref{fig:blackjack2}

\begin{figure}[ht]
\centering
\includegraphics[width=3in]{IMAGE ADDRESS} %NEED IMAGE ADDRESS FROM PROJECT PROPOSAL 
\caption{Concept mockup of the Adafruit CLUE Board screen displaying our Blackjack game.}
\label{fig:blackjack2}
\end{figure}

To effectively produce the handheld blackjack game we will need hardware provided to us. The list of hardware we will require in order to accomplish our task is listed below in Table \ref{table:parts_list}. The arcade buttons are required for the user inputs "hit", "stand", and "split" in order to interact with the game while it is running.

\begin{table}[ht]
\centering
\begin{tabular}{|l|l|l|l|}
    \hline
    \textbf{Part} & \textbf{Quantity} & \textbf{Additional Description} \\ \hline
    Arcade Button & 4 & 1 Red, 1 Green, 2 Blue \\ \hline
    Adafruit CLUE Board & 1 &  \\ \hline
    Battery & 1 & For Clue Board \\ \hline
    Battery Connector & 1 &  \\ \hline
\end{tabular}
\caption{Parts Required for Blackjack Handheld Game}
\label{table:parts_list}
\end{table}


%\begin{lstlisting}[language=Arduino]
%#include <Adafruit_CircuitPlayground.h>

%void setup() {
%  CircuitPlayground.begin();
%}

%void loop() {
%  CircuitPlayground.clearPixels();

% delay(500);

%  CircuitPlayground.setPixelColor(0, 255,   0,   0);
%  CircuitPlayground.setPixelColor(1, 128, 128,   0);
%  CircuitPlayground.setPixelColor(2,   0, 255,   0);
%  CircuitPlayground.setPixelColor(3,   0, 128, 128);
%  CircuitPlayground.setPixelColor(4,   0,   0, 255);
  
%  CircuitPlayground.setPixelColor(5, 0xFF0000);
%  CircuitPlayground.setPixelColor(6, 0x808000);
%  CircuitPlayground.setPixelColor(7, 0x00FF00);
%  CircuitPlayground.setPixelColor(8, 0x008080);
%  CircuitPlayground.setPixelColor(9, 0x0000FF);
 
%  delay(5000);
%}
%\end{lstlisting}

\section{STATUS}


\section{RESULTS}


\section{FUTURE WORK}


\section{CONCLUSION}
%Finally, wrap up your proposal. This only needs to be one or two paragraphs, but it should conclude with what you plan to do and the why and how. Yes, this may seem repetitive, but that is intentional. Do not forget to update your references as those will appear below in a seperate page.

The goal of this project is to create a functioning game of blackjack where the player competes against the computer. This game will use multiple decks of cards in the form of a list in the system. User input will determine if they hold or are given a new random card. This will progress until either the user or the dealer busts at which that point a winner will be declared. 

This game will also utilize basic card counting strategies to help the player compete against the dealer. The purpose of this is to help teach the user how basic card counting works. Since card counting is just statistical analysis of one or many decks of cards, this can also be applied to other card games. Teaching the user how to count cards can help them in the future during other card-based games. 


\newpage
\section{References}
\printbibliography[heading=subbibintoc]
\bibliographystyle{plain}
\bibliography{ref}

@article{zimran2009game,
  title={The Game of Blackjack and Analysis of Counting Cards},
  author={Zimran, Ariell and Klis, Anna and Fuster, Alejandra and Rivelli, Christopher},
  year={2009}
}

@article{crafton2020counting,
  title={Counting Cards: Exploiting Variance and Data Distributions for Robust Compute In-Memory},
  author={Crafton, Brian and Spetalnick, Samuel and Raychowdhury, Arijit},
  journal={arXiv preprint arXiv:2006.03117},
  year={2020}
}

@misc{dukeu, title={Could mental math boost emotional health?}, url={https://www.eurekalert.org/news-releases/820759}, journal={EurekAlert!}, author={DukeU}}
 
@misc{hargroves_2021, title={7 surprising mental health benefits of playing card games}, url={https://thetreatmentspecialist.com/benefits-of-playing-card-games/}, journal={The Treatment Specialist}, author={Hargroves, Philip}, year={2021}, month={Feb}}

\end{document}
